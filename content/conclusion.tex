\section{Conclusions and Future Works}

\subsection{Conclusions}

This is a repeat.
Position figures and tables at the tops and bottoms of columns.
Avoid placing them in the middle of columns. Large figures and tables
may span across both columns. Figure captions should be below the figures;
 table captions should be above the tables. Avoid placing figures and tables
  before their first mention in the text. Use the abbreviation ``Fig. 1'',
  even at the beginning of a sentence.
Figure axis labels are often a source of confusion.
Try to use words rather then symbols. As an example write the quantity ``Inductance",
 or ``Inductance L'', not just.
 Put units in parentheses. Do not label axes only with units.
 In the example, write ``Inductance (mH)'', or ``Inductance L (mH)'', not just ``mH''.
 Do not label axes with the ratio of quantities and units.
 For example, write ``Temperature (K)'', not ``Temperature/K''.


\subsection{Future Works}

This is a repeat.
Position figures and tables at the tops and bottoms of columns.
Avoid placing them in the middle of columns. Large figures and tables
may span across both columns. Figure captions should be below the figures;
 table captions should be above the tables. Avoid placing figures and tables
  before their first mention in the text. Use the abbreviation ``Fig. 1'',
  even at the beginning of a sentence.
Figure axis labels are often a source of confusion.
Try to use words rather then symbols. As an example write the quantity ``Inductance",
 or ``Inductance L'', not just.
 Put units in parentheses. Do not label axes only with units.
 In the example, write ``Inductance (mH)'', or ``Inductance L (mH)'', not just ``mH''.
 Do not label axes with the ratio of quantities and units.
 For example, write ``Temperature (K)'', not ``Temperature/K''. \cite{ctrl:chen-ghwbhs:2010}
