\section{Introduction}
% what why when where who how
The past few years have witnessed unprecedented expansion of commercial computing operations as the idea of cloud computing has become more mainstream and widely adopted by forward thinking technical organizational leadership.  This rate of adoption promises to increase in the near future as well.  With this expansion has come opportunity as well as risk, embodied by recent major service outages at leading cloud providers like Amazon.  These issues promise to become more difficult to control as managed infrastructure expands.  This expansion will simply not be possible without large amounts of automation in all aspects of cloud computing systems.

The current state of the art in cloud systems is poorly differentiated and not as customer-focused as it could be.  Current providers place the responsibility of monitoring performance and proving outages on the consumer rather than providing more transparent and monitorable infrastructure \cite{ctrl:amazon-cloud-watch}.  Furthermore, providers as a whole usually provide one type of service level agreement (SLA) in a loosely-defined one-size-fits-all type of arrangement.  This provides strong differentiating opportunities for smaller, second generation cloud system providers who have established the technology required to scalably manage multiple, competing SLAs on the same infrastructure in tandem with clear customer system visibility.

These second generation providers will rely on automated infrastructure management in order to scale.  One of the first steps toward automating these systems is automating SLA management and compliance.

Herein, we will elaborate the idea of applying usage management to single system governed by multiple different types of SLAs.  We will define the problem, more formally describe SLAs, analyze the implications of that formality, and using this information create a prototypical control system.

In Section \ref{sec:cloud-models}, this paper begins by describing the different types of cloud computing models that generally exist today and how they manage services.  Immediately thereafter, we propose a possible future model in which users can have unique SLAs that more closely fit their needs rather than shoehorning their computing needs into a previously configured contract.  Then, in Section \ref{sec:SLA-defined}, we more formally define an SLA, and show how to convert one to an evaluatable expression.  In the following section, Section \ref{sec:SLA-analysis}, we analyze the new SLA model and extract specific theoretical limits on computability and discuss implications thereof.  Finally in Section \ref{sec:control} we use our new conclusions to design a prototypical control system using these principles.

\subsection{Previous Work}
As cloud computing is emerging as the future of utility systems hosting for consumer-facing applications.  In these kinds of systems, components, applications, and hardware are provided as utilities over the Internet with associated pricing schemes pegged by system demand.  Users accept specific QoS guidelines that providers use to provision and eventually allocate resources. These guidelines become the basis over which providers charge for services.

Over the past few years multiple service-based paradigms like web services, cluster computing and grid computing have contributed to the development of what we now call cloud computing \cite{Bu:09}. Cloud computing distinctly differentiates itself from other service-based computing paradigms via a collective set of distinguishing characteristics:  market orientation, virtualization, dynamic provisioning of resources, and service composition via multiple service providers \cite{BuYeVeBrBr:09}. This implies that in cloud computing, a cloud-service consumer's data and applications reside inside that cloud provider's infrastructure for a finite amount of time.  Partitions of this data can in fact be handled by multiple cloud services, and these partitions may be stored, processed and routed through geographically distributed cloud infrastructures. These activities occur within a cloud, giving the cloud consumer an impression of a single virtual system.  These operational characteristics of cloud computing can raise concerns regarding the manner in which cloud consumer's data and applications are managed within a given cloud. Unlike other computing paradigms with a specific computing task focus, cloud systems enable cloud consumers to host entire applications on the cloud (i.e. Software as a Service) or to compose services from different providers to build a single system. As consumers aggressively start exploiting these advantages to transition IT services to external utility computing systems, the manner in which data and applications are handled within those systems by various cloud services will become a matter of serious concern.

A growing body of research has begun to appear over the past two years applying control theory to tuning computer systems.  These range from controlling network infrastructure \cite{ctrl:ariba-GL:2009} to controlling virtualized infrastructure and specific computer systems \cite{ctrl:wang-cgswrzh:2009}, \cite{ctrl:kjaer-kr:2009} to exploring feedforward solutions based on predictive modeling \cite{ctrl:abdelwahed-bsk:2009}.  Significant open questions remain to research within this field \cite{ctrl:Zhu:2009:CTB:1496909.1496922}, \cite{ctrl:hellerstein-sw:2009}.
