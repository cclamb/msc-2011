\section{Introduction}

Your goal is to simulate, as closely as possible, the usual appearance of typeset
 papers. This document provides an example of the desired layout and contains
 information regarding desktop publishing format, type sizes, and type faces.

\subsection{Full-Size Camera-Ready (CR) Copy}

If you have desktop publishing facilities, (the use of a computer to aid
 in the assembly of words and illustrations on pages) prepare your CR paper
  in full-size format, on paper 21.6 x 27.9 cm (8.5 x 11 in or 51 x 66 picas).
  It must be output on a printer (e.g., laser printer) having 300 dots/in, or
  better, resolution. Lesser quality printers, such as dot matrix printers,
   are not acceptable, as the manuscript will not reproduce the desired quality.

\subsubsection{Typefaces and Sizes:} There are many different typefaces and a large
variety of fonts (a complete set of characters in the same typeface, style,
 and size). Please use a proportional serif typeface such as Times Roman,
 or Dutch. If these are not available to you, use the closest typeface you
  can. The minimum typesize for the body of the text is 10 point. The minimum
  size for applications like table captions, footnotes, and text subscripts
  is 8 point. As an aid in gauging type size, 1 point is about 0.35 mm (1/72in).
   Examples are as follows:

\subsubsection{Format:} In formatting your original 8.5" x 11" page, set top and
bottom margins to 25 mm (1 in or 6 picas), and left and right margins
to about 18 mm (0.7 in or 4 picas). The column width is 88 mm (3.5 in or 21 picas).
 The space between the two columns is 5 mm(0.2 in or 1 pica). Paragraph
 indentation is about 3.5 mm (0.14 in or 1 pica). Left- and right-justify your
 columns. Cut A4 papers to 28 cm. Use either one or two spaces between sections,
 and between text and tables or figures, to adjust the column length.
  On the last page of your paper, try to adjust the lengths of the
  two-columns so that they are the same. Use automatic hyphenation and
   check spelling. Either digitize or paste your figures.

\begin{table}
\caption{An Example of a Table}
\label{table_example}
\begin{center}
\begin{tabular}{|c||c|}
\hline
One & Two\\
\hline
Three & Four\\
\hline
\end{tabular}
\end{center}
\end{table}


%%%%%%%%%%%%%%%%%