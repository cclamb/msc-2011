\section{Cloud System Models}
Current cloud systems do not ignore SLA restrictions; rather, they are designed from the ground up to support a single type of SLA.  That SLA generally encompasses total system uptime and some kind of response time metric \cite{amazon-sla,rackspace-sla}.  If for some reason the cloud provider can no longer adhere to the terms outlined, some kind of compensation strategy applies to affected customers.  Future cloud providers can very well use the ability to support multiple SLAs as a way to differentiate available products from competitors.

\subsection{Current Model}
Current systems like Amazon's EC2 or Rackspace products are designed around high availability, and this is reflected in the focus of their supplied SLAs.  This common design focus is also evident in the artifacts generated by other vendors \cite{google-arch}.  Furthermore, Amazon offers clear guidance on how to develop systems that take advantage of their robust architecture as well as services that provide some measure of automatic scaling \cite{amazon-best-practice,amazon-fault-tolerant}.  This combination of market leading position and products and the extensive supplied guidance make Amazon a clear choice to examine when reflecting on the current state-of-the-art.

Amazon's Cloud Watch products used in tandem with Auto Scaling provide the ability to control the number of deployed instances in response to specific system loads \cite{amazon-cloud-watch,amazon-auto-scale}.  Cloud Watch gives customers the ability to monitor various system performance metrics for their virtual machines, including but not limited to latency, processor use, and request counts.  Furthermore, users can set resource levels at which additional EC2 instances are created or destroyed.  This provides some level of personalized management and control over deployed systems within Amazon's cloud infrastructure.

\subsection{Future Reference Model}

\section{Service Level Agreements Defined}

\section{Controlling with Service Level Agreements}

\subsection{Computational Complexity}

\subsection{Space Complexity}

\subsection{Verification v. Solution}

\subsection{Approximation and other techniques}





